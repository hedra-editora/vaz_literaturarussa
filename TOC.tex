

\thispagestyle{empty}

{\BebasNeue{\LARGE\textbf{SUMÁRIO}}} \\
\\
\\

\footnotesize{
{\MyriadPro{\textbf{[007]}
Nota à edição

\textbf{[009]}
Prefácio \\
}}}

{\BebasNeue{
{\Large\textbf{OBRAS E AUTORES}\\

\large\textbf{ROMANTISMO E REALISMO}
}}}

\smallskip

{\MyriadPro{
\footnotesize{
\textbf{[019]}
Púchkin e o começo da literatura russa

\textbf{[023]}
Púchkin: algumas considerações

\textbf{[025]}
\emph{Evguéni Oniéguin}

\textbf{[031]}
Gógol, o pregador do feudalismo

\textbf{[035]}
Ivan Aleksándrovitch Gontcharóv e seu épico: \emph{Oblómov}

\textbf{[039]}
Ivan Turguêniev e a \emph{intelligentsia} russa

\textbf{[051]}
\emph{Memórias de um caçador}, de Ivan Turguêniev

\textbf{[055]}
O século \scalebox{0.8}{XIX} na Rússia e os ícones da prosa mundial\\
}}}

{\BebasNeue{\large{\textbf{DOSTOIÉVSKI}}}}

\smallskip

{\MyriadPro{
\footnotesize{
\textbf{[063]}
Dostoiévski e Púchkin

\textbf{[071]}
A longa provação de Dostoiévski

\textbf{[075]}
Algumas questões fundamentais na vida e obra de Dostoiévski

\textbf{[085]}
\emph{Os irmãos Karamázov} e as vertentes na obra de Dostoiévski

\textbf{[089]}
A consciência em Dostoiévski (\emph{Crime e castigo})

\textbf{[103]}
De Mikhail Bakhtin a Milan Kundera, os intérpretes de Dostoiévski

\textbf{[107]}
Aurora Bernardini entrevista Joseph Frank\\
}}}


{\BebasNeue{\large{\textbf{TOLSTÓI}}}}

\smallskip

{\MyriadPro{
\footnotesize{
\textbf{[115]}
Tolstói e Dostoiévski

\textbf{[119]}
Considerações à margem de \emph{Anna Kariénina}

\textbf{[125]}
Diários e outros

\textbf{[131]}
As cartilhas do Conde Lev Nikoláievich Tolstói

\textbf{[135]}
\emph{A morte de Ivan Ilitch} -- Lev Tolstói em quadrinhos

\textbf{[137]}
\emph{Khadji-Murát}: um herói fiel a seu código de valores\\
}}}

{\BebasNeue{\large{\textbf{NOVO REALISMO}}}}

\smallskip

{\MyriadPro{
\footnotesize{
\textbf{[141]}
Tchékhov, o intérprete do grande tédio russo

\textbf{[149]}
\emph{O músico cego} e \emph{Em má companhia}, de Vladímir Korolenko

\textbf{[153]}
O evocador de um mundo primitivo -- sobre Górki

\textbf{[157]}
Ivan Búnin e seus \emph{Contos escolhidos}\\
}}}

{\BebasNeue{\large{\textbf{VANGUARDAS E MODERNISMO}}}}

\smallskip

{\MyriadPro{
\footnotesize{
\textbf{[161]}
Traduzir \emph{\scalebox{0.8}{KA}}, de Velimir Khlébnikov

\textbf{[165]}
Encontro de andarilhos: Manoel de Barros e Velimir Khlébnikov

\textbf{[181]}
\emph{O percevejo}, de Vladímir Maiakóvski: o trágico para além do político

\textbf{[185]}
Marina Tsvetáieva, poetisa russa: esboço de vida e obra

\textbf{[199]}
\emph{Encontros com Liz e outras histórias}, de Leonid Dobýtchin

\textbf{[203]}
Víktor Chklóvski: \emph{Viagem sentimental}

\textbf{[207]}
O mundo absurdo de Daniil Kharms\\
}}}


{\BebasNeue{\large{\textbf{CONTEMPORÂNEOS (SÉCULO XX)}}}}
\thispagestyle{empty}
\smallskip

{\MyriadPro{
\footnotesize{
\textbf{[223]}
Vladímir Nabókov: arte duradoura e talento individual

\textbf{[227]}
Sobre Bródski

\textbf{[237]}
Um escritor russo na América: Dovlátov (\emph{Parque Cultural} e \emph{O ofício})

\textbf{[243]}
Friedrich Gorenstein: \emph{Salmo}\\
}}}

{\BebasNeue{\Large{\textbf{TEÓRICOS}}}}

\smallskip

{\MyriadPro{
\footnotesize{
\textbf{[249]}
\emph{Antologia do pensamento crítico russo (1802–1901)}

\textbf{[253]}
Formalismo russo, uma revisão e uma atualização

\textbf{[265]}
Aurora Bernardini entrevista Victor Erlich\\	
}}}

{\BebasNeue{\Large{\textbf{ENTREVISTAS COM AURORA BERNARDINI}}}}

\smallskip

{\MyriadPro{
\footnotesize{
\textbf{[273]}
Entrevista a Belkiss Rabello: sobre Marina Tsvetáieva

\textbf{[283]}
Entrevista à Kalinka\\
}}}

{\BebasNeue{\Large{\textbf{RELATOS}}}}

\smallskip

{\MyriadPro{
\footnotesize{
\textbf{[295]}
Saudação a Boris Schnaiderman

\textbf{[305]}
Minha última viagem sentimental à \scalebox{0.8}{URSS} (1989)
}}}

\medskip

{\BebasNeue{\Large{\textbf{BIBLIOGRAFIA GERAL}}

{\MyriadPro{\footnotesize{\textbf{[324]}}}}
%\medskip

%\Large{\textbf{ÍNDICE ONOMÁSTICO}}

\medskip

\Large{\textbf{SOBRE AUTORA E COLABORADORES}}

{\MyriadPro{\footnotesize{\textbf{[337]}}}}
}}

