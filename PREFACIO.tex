\chapter*{Prefácio\\
\bigskip
\emph{A crítica deve brotar de uma dívida de amor}}

\addcontentsline{toc}{part}{Prefácio}

\hedramarkboth{Prefácio}{}


Com a afirmação que intitula esse prefácio George Steiner abre o seu primeiro livro de crítica
literária, de 1959, intitulado \emph{Tolstói ou Dostoiévski -- Um ensaio
sobre o velho criticismo}\footnote{STEINER, G. \emph{Tolstói ou
  Dostoiévski -- Um ensaio sobre o velho criticismo}, São Paulo:
  Perspectiva, 2017.}. Através de algum instinto primário de comunhão,
pondera o crítico, buscamos passar aos outros a qualidade e a força de
nossa experiência. ``Gostaríamos de persuadi"-los a se abrirem para ela.
Dessa tentativa de persuasão se originam as intuições mais verdadeiras
da crítica''.\footnote{Idem.ibidem, pág.1.}

Estas \emph{Aulas de literatura russa}, de autoria de Aurora Fornoni
Bernardini, constituem, certamente, o fruto de suas intuições mais
verdadeiras, surgidas e acumuladas no decorrer de vários anos no
exercício ininterrupto da pesquisa acadêmica, da crítica literária e da
docência no campo dos estudos russos.

Na verdade, esta coletânea de ensaios vem preencher entre nós muito
oportunamente uma lacuna editorial: embora a literatura e a cultura
russas vivam hoje no Brasil um de seus momentos talvez mais profícuos de
difusão e divulgação, a considerar o número cada vez maior de títulos e
autores russos traduzidos e disponíveis em língua portuguesa nas
livrarias brasileiras, o leitor não dispõe, no entanto, de uma escolha
tão ampla quando se trata de estudos ensaísticos de crítica literária,
produzidos por nossa eslavística e dedicados, em particular, à análise e
à teoria literária.

A presente antologia estruturada em duas partes (``Estudos sobre
autores'' e ``Estudos teóricos'') constrói um movimento crítico de viés,
em certa medida, historiográfico e cronológico da literatura russa
(especialmente na abordagem dos autores constantes da primeira parte),
sem prejuízo algum da verticalidade, que marca as análises e
interpretações de obras literárias, de fatos artísticos, filosóficos,
sociológicos, linguísticos, culturais e históricos, embasadas em agudo
senso de pesquisa e ensino acadêmicos.

Não sem razão: Aurora Fornoni Bernardini é professora, ensaísta,
escritora, tradutora e artista plástica e tem papel relevante na
tradição dos estudos da literatura russa e da tradução literária de
textos russos no Brasil. Na Universidade de São Paulo criou nos anos de
1960, juntamente com Boris Schnaiderman, um núcleo de ensino e pesquisa,
ainda em plena expansão, que se tornou referência na difusão das
primeiras traduções diretas de textos não apenas de titãs da prosa e da
poesia russas, mas também de teóricos russos fundamentais para a crítica
e a teoria literárias, que alimentaram em nosso meio, e até hoje
alimentam, estudiosos da literatura e da teoria literária (e não apenas
a russa).

Basta citar alguns dos muitos autores traduzidos por Aurora Bernardini
no âmbito teórico e literário (Tchekhov, Akhmátova, Tyniánov,
Meletínski, Bakhtin, Eisenstein, Todorov, Marina Tsvetáieva, Bábel,
Khlébnikov, Turguêniev, Leskov, Mandelstam, Sologúb\ldots{}) para se ter uma
ideia do largo espectro de seus interesses intelectuais e de sua
contribuição para o alargamento do horizonte intelectual e cultural de
nosso país. A lista seria ainda maior se a ela acrescentássemos autores
e títulos italianos e ingleses, a que o público brasileiro teve acesso
em traduções inéditas.

Numa época em que a especialização crescente grassa o ambiente acadêmico
e intelectual, as atividades de crítica e de criação de Aurora
Bernardini e, sobretudo, as suas preocupações pedagógicas e de formação
vêm demonstrar o alcance de seu pensamento e da missão de seu ofício
como docente e intelectual.

Prova disso encontramos nessas \emph{Aulas de literatura russa} --- o título
não poderia ser mais adequado.

O conjunto de textos aqui reunidos configura menos uma antologia de
ensaios esparsos, e mais, isto sim, uma obra, à qual não faltam unidade
e organicidade, posto que resultado de apurada reflexão, amparada por
sólido aparato teórico"-crítico, explicitado em muitos dos ensaios que
compõem a segunda parte deste volume, e nutrida, como a autora informa
em nota introdutória, pelas inúmeras aulas, artigos, palestras, resenhas
e diferentes publicações produzidas durante um período de sua carreira
acadêmica, e cuja pertinência e importância para os estudos russos podem
ser agora melhor aferidas numa leitura integrada.

Nesse sentido, a segunda parte dedicada aos ``Estudos teóricos''
conforma, por assim dizer, algumas das chaves metodológicas e
elucidativas, a instaurar um diálogo subliminar com os textos que
estruturam a primeira parte do livro. Esse diálogo crítico"-teórico entre
os ensaios do livro propicia uma espécie de ressonância estética e
histórico"-cultural entre os autores e textos literários analisados, e
encaminha, afinal, uma apreensão dos elos explícitos ou implícitos que
impulsionam o desenvolvimento da literatura e da cultura russas ao longo do
tempo.

As ``aulas'' constantes da primeira parte do volume, dedicadas a
Púchkin, Gógol, Gontcharóv, Turgueniev, Dostoiévski, Tolstói, Tchékhov,
Búnin, Górki, Bródski, Nabókov, Kharms\ldots{} e a outros nomes"-chaves da
história literária e cultural russa, desde a sua formação até os dias de
hoje, empreendem recortes analíticos argutos e originais, sempre
balizados, em grande maioria, por análises e interpretações coladas às
diferentes escritas e textualidades na busca de uma leitura imanente do
texto literário, sem desdenhar, porém, um olhar crítico transversal para
a captação de aspectos biográficos, filosóficos, políticos, sociais ou
ideológicos da criação artística.

Tal \emph{modus operandi} se alicerça, certamente, na autoridade de quem
acompanhou a introdução das teorias do formalismo russo no ambiente
intelectual e acadêmico brasileiro. No ensaio intitulado ``Formalismo
russo, uma revisão e uma atualização'' e no texto"-entrevista a seguir
(``Reverberações do formalismo russo na crítica literária americana''),
ambos integrantes da segunda parte do volume, a autora discorre sobre as
inflexões desse movimento teórico na teoria literária contemporânea,
tema de uma alentada pesquisa realizada nos idos de 1990. Ao revisar e
passar a limpo nomes seminais da teoria literária russa como Iúri
Tyniánov, Roman Jakobson, Viktor Chklóvski, Óssip Brik, Tomachévski,
Boris Eikhenbaum, Vladímir Propp e outros, Aurora Bernardini ressalta a
vigência de conceituações essenciais do formalismo russo, capazes de
responder a questões que cercam a pós"-modernidade: ``o que é
literatura?'', ``o que diferencia a literatura de outros domínios da
escrita?'', ``como se estrutura o mundo do texto frente ao mundo de que
ele é imagem?''.

O conceito de ``dominante'' como princípio organizador do texto, a
questão das funções da linguagem, a da equivalência em poesia dos dois
eixos, o paradigmático (metafórico) e o da contiguidade (metonímico), o
conceito de ``estranhamento'', a questão da determinação da ``diferença
específica'', do traço distintivo, do critério qualitativo que permite,
no caso da literatura, estabelecer seus limites frente às outras
expressões das Humanidades, enfim, a análise dos procedimentos, que
implicam, afinal, o conceito de ``literaturnost'' (literariedade), caro
aos formalistas russos, tudo isso está posto aqui sob exame para atestar
a permanência dessas teorias e sua eficácia para a abordagem do fato
literário e do fato artístico, mesmo tendo passada a voga dos anos de
1960--1970.

Exemplo disso, e da já mencionada simbiose teórico"-crítica entre as duas
partes do livro, são as análises das poéticas de Khlébnikov, Tsvetáieva,
Bródski, Kharms e Manoel de Barros apresentadas na seção ``Poesia''. Os
poetas e seus respectivos poemas são perscrutados, mesmo que de modo
oblíquo, à luz de muitas das considerações contidas no ensaio
``Formalismo e poesia: um balanço da contribuição dos formalistas
russos numa área específica da investigação literária'', em que a
concepção da linguagem poética como discurso autônomo e como uma
dinâmica semântica específica aparece explicitada na práxis analítica da
autora. Reverberam nessas análises, por exemplo, as lições de I.
Tyniánov captadas em \emph{O problema da linguagem poética} e
esmiuçadas com profundidade no ensaio referido.

Ao rebater com veemência a apreensão distorcida de certa crítica
detratora do formalismo russo, movida, segundo a autora, por um
conhecimento muitas vezes superficial, textos copilados e mal
traduzidos, esses ensaios teóricos desenham um amplo painel do movimento
russo desde os seus precursores, seu surgimento e desenvolvimento, e o
legado para a crítica e a teoria literária contemporâneas. Iluminam"-se,
assim, os eixos essenciais por meio dos quais os formalistas russos
puderam compreender o texto literário como um ``dinamismo interno'' de
um determinado sistema, com suas leis imanentes, inserindo"-o, ao mesmo
tempo, nas diferentes séries sociais e históricas, aspecto este, pouco
relevado por uma crítica mais apressada, mas sublinhado em ambas as
partes deste volume de ensaios.

Há que se ressaltar que a lucidez crítica esboçada nesses ``Estudos
teóricos'' afunda raízes em três importantes trabalhos acadêmicos
escritos pela autora em diferentes momentos de sua trajetória crítica:
\emph{Materiais para o estudo do futurismo italiano e do futurismo russo}
(1970), \emph{Poesia e poéticas do futurismo russo e italiano} (1973) e
\emph{Indícios flutuantes em Marina Tsvetáieva} (1977).

A esse debate instaurado no livro sobre o formalismo russo e a
contemporaneidade participa ninguém menos do que Victor Erlich, autor de
um dos mais importantes estudos sobre o tema (\emph{Russian formalism:
history -- doctrine}), a quem a autora recorre por ocasião de uma visita
a Yale, deixando registrada na instigante entrevista publicada nessa
coletânea a profética asserção do eminente estudioso, discípulo de Roman
Jakobson: ``após o formalismo russo nada de mais original ou importante
teria surgido no domínio da Teoria da Literatura''.

Aliás, a estratégia crítica de dar viva voz a especialistas renomados
para a discussão de temas, autores e obras específicas se mostra
produtiva em outra entrevista aqui incluída. O diálogo com Joseph Frank,
um dos maiores conhecedores da vida e obra de F. Dostoiévski, vem
esclarecer não apenas questões atinentes ao contexto estético,
ideológico e religioso, a que o romancista russo reage por meio de sua
obra, mas também vieses da crítica dostoievskiana, como as teorias de
Bakhtin, por exemplo.

Corrobora esse incessante movimento de interlocução crítica presente na
coletânea uma outra vertente analítica: a literatura comparada. Vários
são os ensaios em que os estudos comparados tecem aproximações e
conexões surpreendentes: ``Encontro de andarilhos'', ``De Tchekhov a
Pirandello'' e ``Aspectos da natureza em Velimir Khlébnikov e em Manoel
de Barros'' propõem imersões comparativas desafiadoras entre universos
artísticos e culturais, \emph{a priori}, dessemelhantes. Mas a crítica
comparada se insinua também em ``Dostoiévski e Púchkin'' e ``Tolstói e
Dostoiévski'', desta feita numa espécie de interação russa
``intramuros''.

Resta ainda salientar outros desdobramentos dialógicos. Como a perseguir
a herança teórica dos formalistas russos, os ensaios ``Dos diários de
Serguei Eisenstein e outros ensaios'', ``O orgânico e o patético em S. M.
Eisenstein'' e ``O papel do conflito no conto russo de magia'' aludem a
aspectos relevantes das pesquisas mais avançadas nos campos da
Antropologia Cultural e da Semiótica da Cultura. Semioticistas russos de
primeira grandeza, como V. V. Ivánov, Tóporov, Bogatyrióv, Meletínski e
Lótman se refratam na leitura integradora das ideias teóricas e da
cinematografia de Eisenstein, em que a linguagem do cinema se articula a
outros sistemas de signos. Ficam destacadas, assim, as relações entre os
fenômenos artísticos e os mecanismos da cultura e da história, como bem
estabelece Iúri Lotman ao tecer as necessárias correlações entre a
semiótica da arte e a semiótica da cultura, procedimento, aliás,
empregado pela autora em vários estudos dedicados a prosadores e poetas
russos de diferentes momentos da História literária russa.

Desta recolha de textos de Aurora Bernardini, sob organização de um de
seus discípulos, depreende"-se, afinal, um dos axiomas fundamentais dos
estudos literários, qual seja, a concepção da crítica literária não
apenas como uma espécie de apêndice superficial da literatura, mas,
conforme salienta Todorov, como seu duplo necessário, porque um texto
artístico talvez nunca possa dizer a totalidade da sua verdade: ele não
deve significar, mas simplesmente ser uma nova luz lançada sobre o
mundo.

Conclui esse amplo percurso crítico uma inesperada e subjetiva nota
``sentimental''. ``Minha última viagem sentimental à \versal{URSS}'' é um relato
de viagem colorido por impressões pessoais e recordações, que evocam um
tempo vivido numa Rússia pretérita e uma experiência afetiva presente na
memória do observador. Permanecem o mesmo olhar e o mesmo rigor
inquiridores na sondagem de uma cultura e de uma história, às quais uma
vida inteira está dedicada. Mas, agora, nessa escrita do eu a crítica se
reveste de criação e a dívida de amor parece brotar dessas derradeiras
linhas\ldots{}

Eis nas páginas que se seguem um convite ao leitor para perfazer essa
bela aventura do espírito.

\begin{flushright}
\emph{Arlete Orlando Cavaliere}
\end{flushright}


