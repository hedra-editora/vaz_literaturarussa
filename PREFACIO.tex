\chapter*{Prefácio\\
\bigskip
\emph{A crítica deve brotar de uma dívida de amor}}

\addcontentsline{toc}{part}{Prefácio}

\hedramarkboth{Prefácio}{}


Com a afirmação que intitula este prefácio George Steiner abre o seu primeiro livro de crítica
literária, de 1959, intitulado \emph{Tolstói ou Dostoiévski -- Um ensaio
sobre o velho criticismo}\footnote{\versal{STEINER}, G. \emph{Tolstói ou
  Dostoiévski -- um ensaio sobre o velho criticismo}. São Paulo:
  Perspectiva, 2017.}. Através de algum instinto primário de comunhão,
pondera o crítico, buscamos passar aos outros a qualidade e a força de
nossa experiência. ``Gostaríamos de persuadi"-los a se abrirem para ela.
Dessa tentativa de persuasão se originam as intuições mais verdadeiras
da crítica'' (\versal{STEINER}, 2017, p. 1).

Estas \emph{Aulas de literatura russa}, de autoria de Aurora Fornoni
Bernardini, constituem, certamente, o fruto de suas intuições mais
verdadeiras, surgidas e acumuladas no decorrer de vários anos no
exercício ininterrupto da pesquisa acadêmica, da crítica literária e da
docência no campo dos estudos russos.

Na verdade, esta coletânea de ensaios vem preencher entre nós muito
oportunamente uma lacuna editorial: embora a literatura e a cultura
russas vivam hoje no Brasil um de seus momentos mais profícuos de
difusão, a considerar o número cada vez maior de títulos e
autores russos traduzidos para a língua portuguesa e disponíveis nas
livrarias brasileiras, o leitor não dispõe de uma escolha
tão ampla quando se trata de estudos ensaísticos,
produzidos por nossa eslavística e dedicados, em particular, à análise e
à teoria literária.

A presente antologia estruturada em quatro partes (``Obras e
autores'', ``Teóricos'', ``Entrevistas'' e ``Relatos'') constrói um movimento crítico de viés,
em certa medida, historiográfico e cronológico da literatura russa
(especialmente na abordagem dos autores da primeira parte),
sem prejuízo algum da verticalidade que marca as análises de obras literárias, de fatos artísticos, filosóficos,
sociológicos, linguísticos, culturais e históricos, embasadas em agudo
senso de pesquisa e ensino acadêmicos.

Não sem razão: Aurora Fornoni Bernardini é professora, ensaísta,
escritora, tradutora e artista plástica e tem papel relevante na
tradição dos estudos e da tradução literária de
textos russos no Brasil. Na Universidade de São Paulo criou nos anos
1960, juntamente com Boris Schnaiderman, um núcleo de ensino e pesquisa,
ainda em plena expansão, que se tornou referência na divulgação das
primeiras traduções diretas de obras russas não apenas de titãs da prosa e da
poesia, mas também de teóricos que alimentaram, e até hoje
alimentam, estudiosos da literatura e da teoria literária (e não apenas
a russa).

Basta citar alguns dos muitos autores traduzidos por Aurora Bernardini
no âmbito teórico e literário --- Tchékhov, Akhmátova, Tyniánov,
Meletínski, Bakhtin, Eisenstein, Tsvetáieva, Bábel,
Khlébnikov, Turguêniev, Ivanóv, Mandelstam, Sologub\ldots{} --- para se ter uma
ideia do largo espectro de seus interesses intelectuais e de sua
contribuição para o alargamento do horizonte cultural de
nosso país. A lista seria ainda maior se a ela acrescentássemos os autores
e títulos italianos e ingleses a que o público brasileiro teve acesso
por meio de suas traduções.

Numa época em que a especialização crescente grassa no ambiente acadêmico, as atividades de crítica e de criação de Aurora
Bernardini e, sobretudo, as suas preocupações pedagógicas e de formação
vêm demonstrar o alcance de seu pensamento e da missão de seu ofício
como docente e intelectual.

Prova disso encontramos nessas \emph{Aulas de literatura russa} --- o título
não poderia ser mais adequado.

O conjunto de textos aqui reunidos configura menos uma antologia de
ensaios esparsos, e mais, isto sim, uma obra à qual não faltam unidade
e organicidade, posto que resultado de apurada reflexão, amparada por
sólido aparato teórico"-crítico e nutrida por inúmeras aulas, artigos, palestras, resenhas
e diferentes publicações, trabalhos cuja pertinência e importância para os estudos russos podem
ser agora melhor aferidas numa leitura integrada.

Nesse sentido, a segunda parte do livro, dedicada a estudos teóricos,
conforma, por assim dizer, algumas chaves metodológicas e
elucidativas que dialogam subliminarmente com os textos que
estruturam a primeira parte. Esse diálogo crítico"-teórico entre
os ensaios do livro propicia uma espécie de ressonância estética e
histórico"-cultural entre os autores e textos analisados, e
encaminha, afinal, uma apreensão dos elos explícitos ou implícitos que
impulsionam o desenvolvimento da literatura e da cultura russas ao longo do
tempo.

As ``aulas'' constantes da primeira parte do volume --- dedicadas a
Púchkin, Gógol, Gontcharóv, Turguêniev, Dostoiévski, Tolstói, Tchékhov,
Búnin, Górki, Maiakóvski, Tsvetáieva, Kharms, Bródski, Nabókov\ldots{} e a outros nomes"-chaves da
história literária e cultural russa, desde a sua formação até os dias de
hoje --- empreendem recortes analíticos argutos e originais, balizados pela busca de uma leitura imanente do
texto literário, sem desdenhar, porém, um olhar crítico transversal para
a captação de aspectos biográficos, filosóficos, políticos, sociais ou
ideológicos da criação artística.

Tal \emph{modus operandi} se alicerça, certamente, na autoridade de quem
acompanhou a introdução das teorias do formalismo russo no ambiente
intelectual brasileiro. No ensaio intitulado ``Formalismo
russo, uma revisão e uma atualização'', a autora discorre sobre as
inflexões desse movimento na teoria literária contemporânea,
tema de uma alentada pesquisa realizada nos idos de 1990. Ao
passar a limpo nomes seminais da teoria literária russa, como Iúri
Tyniánov, Roman Jakobson, Víktor Chklóvski, Óssip Brik, Boris Tomachévski,
Boris Eikhenbaum, Vladímir Propp e outros, Aurora Bernardini ressalta a
vigência de conceituações essenciais do formalismo russo, capazes de
responder a questões que cercam a pós"-modernidade: ``O que é
literatura?'', ``O que diferencia a literatura de outros domínios da
escrita?'', ``Como se estrutura o mundo do texto frente ao mundo de que
ele é imagem?''.

O conceito de \emph{dominante} como princípio organizador do texto; as
funções da linguagem; a equivalência em poesia de dois
eixos, o paradigmático (metafórico) e o da contiguidade (metonímico); o
conceito de \emph{estranhamento}; a questão da determinação da ``diferença
específica'', do traço distintivo, do critério qualitativo que permite estabelecer os limites da literatura frente às outras
expressões das Humanidades; enfim, a análise dos procedimentos que
implicam, afinal, o conceito de \emph{literaturnost} (\emph{literariedade}), caro
aos formalistas russos, está posta aqui sob exame para atestar
a permanência dessas teorias e sua eficácia na abordagem do fato
literário e artístico.

Exemplo disso, e da já mencionada simbiose teórico"-crítica entre as duas
primeiras partes do livro, são as análises das poéticas de Khlébnikov, Tsvetáieva,
Kharms e Bródski apresentadas nas seções ``Vanguardas e modernismo'' e ``Contemporâneos (século \versal{XX})''. Os
poetas e seus respectivos poemas são perscrutados, mesmo que de modo
oblíquo, à luz de muitas das considerações contidas no ensaio
sobre o formalismo russo. Dessa maneira, a
concepção da linguagem poética como discurso autônomo e como uma
dinâmica semântica específica aparece explicitada na práxis analítica da ensaísta.

Ao rebater com veemência a apreensão distorcida de certa crítica
detratora do formalismo russo, movida por um
conhecimento muitas vezes ``superficial, textos copilados e mal
traduzidos'', Aurora Bernardini nos oferece no referido ensaio um amplo painel desse movimento
russo --- desde seus precursores, seu surgimento e desenvolvimento, até o
legado para a crítica e a teoria literária contemporâneas. Iluminam"-se,
assim, os eixos principais por meio dos quais os formalistas
puderam compreender a obra literária como um ``dinamismo interno'' de
determinado sistema, com suas leis imanentes, inserindo"-a, ao mesmo
tempo, nas diferentes séries sociais e históricas, aspecto este pouco
relevado por uma crítica mais apressada, mas sublinhado neste volume.

Há que se ressaltar que a lucidez crítica esboçada aqui
afunda raízes em três importantes trabalhos acadêmicos
escritos pela autora em diferentes momentos de sua trajetória crítica:
\emph{Materiais para o estudo do futurismo italiano e do futurismo russo}
(1970), \emph{Poéticas do futurismo russo e italiano} (1973) e
\emph{Indícios flutuantes em Marina Tsvetáieva} (1977).

Desse debate sobre o formalismo russo e a
contemporaneidade participa ninguém menos do que Victor Erlich, autor de
um dos mais importantes estudos sobre o tema,\footnote{\versal{ERLICH}, Victor. \emph{Russian
 formalism: history -- doctrine.} New Haven: Yale University Press,
 1981, 3ª ed.} a quem a autora recorre por ocasião de uma visita
a Yale, deixando registrada na instigante entrevista publicada nesta
coletânea (``Reverberações do formalismo russo na crítica literária 
americana'') a profética asserção do eminente estudioso, discípulo de Roman
Jakobson: ``após o formalismo russo nada de mais original ou importante
teria surgido no domínio da Teoria da Literatura''.

Aliás, a estratégia crítica de dar viva voz a especialistas renomados
para a discussão de temas, autores e obras específicas se mostra
produtiva em outra entrevista aqui incluída. O diálogo com Joseph Frank,
um dos maiores conhecedores da vida e obra de F. Dostoiévski, vem
esclarecer não apenas questões atinentes ao contexto estético,
ideológico e religioso a que o romancista russo reage por meio de sua
obra, mas também vieses da crítica dostoievskiana, como as teorias de
Bakhtin, por exemplo.

Corrobora esse incessante movimento de interlocução crítica presente na
coletânea outra vertente analítica: a literatura comparada. O
ensaio ``Encontro de andarilhos'', entre Velimir Khlébnikov e Manoel
de Barros, propõe imersões comparativas desafiadoras entre universos
artísticos e culturais, \emph{a priori}, dessemelhantes. Mas a crítica
comparada se insinua também em ``Dostoiévski e Púchkin'' e ``Tolstói e
Dostoiévski'', numa espécie de interação russa
``intramuros''.

Desta recolha de textos de Aurora Bernardini, sob organização de dois de
seus discípulos, depreende"-se, afinal, um dos axiomas fundamentais dos
estudos literários --- a concepção da crítica literária não
apenas como uma espécie de apêndice superficial da literatura, mas,
conforme salienta Todorov, como seu duplo necessário, porque um texto
artístico talvez nunca possa dizer a totalidade da sua verdade: ele não
deve significar, mas simplesmente ser uma nova luz lançada sobre o
mundo.

Concluem esse amplo percurso crítico a seção de entrevistas com a
 autora e a de relatos. Além de um valioso testemunho sobre a 
trajetória intelectual do professor, ensaísta e tradudor Boris 
Schnaiderman, a escritora nos apresenta uma inesperada e subjetiva nota
``sentimental''. ``Minha última viagem sentimental à \versal{URSS}'' é um relato
de viagem colorido por impressões e recordações pessoais que evocam um
tempo vivido numa Rússia pretérita presentificado na
memória do observador. Permanecem o mesmo olhar e o mesmo rigor
inquiridores na sondagem de uma cultura e de uma história às quais uma
vida inteira está dedicada. Mas, agora, nessa escrita do eu a crítica se
reveste de criação, e a dívida de amor parece brotar dessas derradeiras
linhas\ldots{}

As páginas que se seguem são um convite ao leitor para perfazer essa
bela aventura do espírito.

\begin{flushright}
\emph{Arlete Orlando Cavaliere}
\end{flushright}


