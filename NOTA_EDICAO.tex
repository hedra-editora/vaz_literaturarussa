\chapter{Nota à edição}

Os textos que compõem o presente livro são baseados nas aulas
ministradas por Aurora Fornoni Bernardini no curso de russo da
Universidade de São Paulo desde 1969, sintetizadas em resenhas, artigos
e ensaios, dos quais foram selecionados principalmente os publicados a partir da década
de 1980 em diversos jornais, livros e revistas brasileiros.

Esta coletânea não tem naturalmente a pretensão de ser um manual
da literatura russa, mas não deixa de cumprir de algum modo
esse papel, visto que abarca períodos, escritores e conceitos da
cultura russa. Os textos, portanto, não estão dispostos conforme as
datas de publicação, mas estão distribuídos em grandes momentos
literários. Assim, a seção que abre o livro, ``Obras e Autores'', é
constituída pelos seguintes tópicos: ``Romantismo e Realismo'',
``Dostoiévski'', ``Tolstói'', ``Novo Realismo'', ``Vanguardas e
Modernismo'', ``Contemporâneos (século \scalebox{0.8}{XX})''. Embora, pela própria
natureza de uma antologia, muitos autores tenham ficado de fora,
criou-se um panorama abrangente, com poéticas analisadas em textos de
dimensões e formatos variados (resenhas, ensaios e capítulos de livros),
revelando a versatilidade intelectual da professora, tradutora e
escritora Aurora Bernardini.

Escritos ao longo de mais trinta anos, os artigos foram revisados,
atualizados, padronizados e, em alguns casos, reunidos e ampliados para
esta edição. Além dos trabalhos publicados em prestigiosos periódicos e
jornais brasileiros (sempre referenciados em notas de rodapé), o livro
traz textos inéditos (sobre Ivan Gontcharóv e Ióssif Bródski).

Para completar esta homenagem ao percurso de Aurora Fornoni Bernardini,
expoente da difusão das letras russas no Brasil, foram selecionados,
afora alguns ensaios sobre teóricos russos, entrevistas com a autora e
relatos pessoais: estes sobre o papel precursor de Boris Schnaiderman nessa
difusão, e sobre a própria experiência da escritora na União Soviética.

Quanto à transliteração dos nomes russos, seguiram"-se as normas
utilizadas pelo curso de russo da \scalebox{0.8}{USP}, salvo alguns nomes próprios de
grafia já consagrada no Brasil.

As notas de rodapé são da autora com algumas colaborações da organização
e da edição, nesse caso assinaladas (\scalebox{0.8}{N}. da \scalebox{0.8}{E}.). As referências
bibliográficas são mencionadas pontualmente, na primeira ocorrência em
nota de rodapé e depois retomadas no corpo de texto (nome do autor e
data da publicação), e também reunidas no fim do livro.

\emph{Aulas de literatura russa: de Púchkin a Gorenstein} foi
organizado por Daniela Mountian e Valteir Vaz e prefaciado por Arlete
Orlando Cavaliere.
